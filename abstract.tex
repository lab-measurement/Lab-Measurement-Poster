\documentclass{scdpg}
\begin{document}
\scBookLanguage{de}
\begin{scAbstract}
%\scNoUseTeX
\scLanguage{en}
\scTitle{Lab::Measurement -- measurement control with Perl}

\scAuthor{*}{Simon}{Reinhardt}{1}
\scAuthor{}{Charles E.}{Lane}{2}
\scAuthor{}{Christian}{Butschkow}{1}
\scAuthor{}{Alexei}{Iankilevitch}{1}
\scAuthor{}{Alois}{Dirnaichner}{1}
\scAuthor{}{Andreas K.}{H\"{u}ttel}{1}

\scAffiliation{1}{Institute for Experimental and Applied Physics, Universit\"{a}t Regensburg, Regensburg, Germany}
\scAffiliation{2}{Department of Physics, Drexel University, Philadelphia, USA}
\scBeginText

\texttt{Lab::Measurement} is a collection of Perl 5 modules providing control of test 
and
measurement devices. It allows for quickly setting up varying and evolving
complex measurement tasks with diverse hardware. Instruments can be connected 
by means such as GPIB (IEEE 488.2), USB-TMC, or VXI-11 / raw network sockets 
via
Ethernet. Internally, third-party backends as e.g. Linux-GPIB, National 
Instruments' NI-VISA library or Zurich Instruments' LabOne API are used, as 
well
as lightweight drivers for USB and TCP/IP-based protocols. The wide range of 
supported connection backends enables cross-platform portability. Dedicated 
instrument driver classes relieve the user from taking care of internal or 
vendor-specific details. A high-level layer provides fast and flexible creation
of nested measurement loops, where e.g. several input variables are varied and
output data is logged into a customizable folder
structure. \texttt{Lab::Measurement} has 
already been successfully used in several low temperature transport 
spectroscopy
setups. It is free software and available at \texttt{http://www.labmeasurement.de/}

\scEndText
\scConference{Berlin 2018}
\scPart{TT}
\scContributionType{Poster}
\scTopic{Low Dimensional Systems: Other Topics}
\scEmail{simon.reinhardt@stud.uni-regensburg.de}
\scCountry{Germany}
\end{scAbstract}
\end{document}
